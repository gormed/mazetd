%%%%%%%%%%%%%%%%%%%%%%%%%%%%%%%%%%%%%%%
%                                                                              
% 		Anforderungen		      		
%                                                                              
%%%%%%%%%%%%%%%%%%%%%%%%%%%%%%%%%%%%%%%

\subsection{Funktionale Anforderungen}
Beschreibung des Softwareprodukts aus Auftraggebersicht und auf oberster Abstraktionsebene; keine Detailbeschreibungen!

Verwenden Sie die folgenden Kürzel, um Ihre Anforderungen eindeutig zu identifizieren.
/PF10/ für die erste funktionale Anforderung
/PF20/ für die zweite funktionale Anforderung
usw.

\subsection{Nicht-Funktionale Anforderungen}

Qualitätsanforderungen
Qualitätsziele anhand einer Tabelle bestimmen, wie unten angeführt:\\

         TODO Tabelle mit Rahmen recherchieren\\
         \ \\
        \begin{tabularx}{\textwidth}{ l l l l l }
            \emph{Systemqualität} & \emph{Sehr gut} & \emph{Gut}& \emph{Normal} & \emph{Nicht relevant}\\
            
			Funktionalität	& & X & &\\
			
			Zuverlässigkeit	& X & & &\\

			Benutzbarkeit	& & X & &\\

			Effizienz		& & & X &\\

			Wartbarkeit		& & & X &\\

			Portabilität		& & & X &\\

        \end{tabularx}
           
\ \\
Tabelle 1: Qualitätsanforderungen\\

Die in Tabelle 1 angegebenen Bewertungen sind nur beispielhaft gewählt.
Eine Verfeinerung der in der Tabelle genannten Qualitätsmerkmale finden sich in der ISO/IEC 9126-1. Je nach Größe des Projekts können Sie mit der o.g. Tabelle arbeiten oder Verfeinerungen angeben.

Verwenden Sie z.B. die folgenden Kürzel, um Ihre Qualitätsanforderungen eindeutig zu identifizieren:
\begin{itemize}
	\item /PQBE10/ für die erste Qualitätsanforderung zur Benutzbarkeit (Erlernbarkeit)
	\item[]für die erste Qualitätsanforderung zur Wartbarkeit (Stabilität)
\end{itemize}
